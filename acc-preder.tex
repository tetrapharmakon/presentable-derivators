\documentclass[a4paper,11pt]{amsart}

\usepackage{
     amsmath
    ,amsfonts
    ,amssymb
    ,blindtext
    ,enumitem
    ,stmaryrd
    ,tikz
}

\usepackage[OT2,T1]{fontenc}
\usepackage[utf8]{inputenc}
\usepackage[russian,english]{babel}
\usepackage{
   amsmath
  ,amsfonts
  ,amssymb
  ,amsthm
  ,leftidx
  ,cite
  ,datetime
  ,enumitem
  ,stmaryrd
  ,todonotes
  ,xspace
  ,xparse
  ,hyphenat
  ,epigraph
  ,kodi
  ,faktor
  ,mathtools
  ,fancyhdr
  ,blindtext
  ,tikz
  ,tikz-cd
  ,wasysym
  ,marginnote
  % ,fdsymbol
}
\usepackage[all,cmtip,2cell]{xy}\UseAllTwocells

\usepackage[cal=pxtx]{mathalfa}
\usepackage{hyperref}
\usepackage{cjhebrew}

%===================[ squares and stuph ]
\usepackage{tikz,tikz-cd}
\usetikzlibrary{arrows}
\usetikzlibrary{babel}

\tikzset{%
implies/.style={double,double equal sign distance,-implies},
shorten <>/.style={shorten >=#1,shorten <=#1}}

\newlength{\spacing}
\setlength{\spacing}{0pt}
%
\def\scaling{.2}
%
\newlength{\raising}
\setlength{\raising}{.5pt}

\def\drawboxvoid{
\begin{tikzpicture}[scale=\scaling]
\draw (0,0) rectangle (1,1);%
\end{tikzpicture}}
%%%%%
\def\drawluangle{\begin{tikzpicture}[scale=\scaling]
\draw (0,0) -- (0,1) -- (1,1);%
\end{tikzpicture}}
%%%%%
\def\drawdbluangle{\begin{tikzpicture}[scale=\scaling]
\draw (0,0) -- (0,1) -- (1,1);%
\end{tikzpicture}}
%%%%%
\def\drawrdangle{\begin{tikzpicture}[scale=\scaling]
\draw (0,0) -- (1,0) -- (1,1);%
\end{tikzpicture}}
%%%%%
\def\drawtwoboxes{\begin{tikzpicture}[scale=\scaling]
\draw (0,0) -- (1,0) -- (1,1) -- (0,1) -- cycle;%
\draw[xshift=1cm] (0,0) -- (1,0) -- (1,1) -- (0,1) -- cycle;%
\end{tikzpicture}}
%%%%%
\def\drawldangle{\begin{tikzpicture}[scale=\scaling]
\draw (1,0) -- (0,0) -- (0,1);%
\end{tikzpicture}}
%%%%%
\def\drawboxvoidbar{\begin{tikzpicture}[scale=\scaling]
\draw (1,1) -- (1,0) -- (0,0) -- (0,1) -- (2,1);%
\end{tikzpicture}}
%%%%%
\def\boxvoid   {\raisebox{\raising}{\drawboxvoid}}
\def\twoboxes  {\raisebox{\raising}{\drawtwoboxes}}
\def\ldangle   {\raisebox{\raising}{\drawldangle}}
\def\boxvoidbar{\raisebox{\raising}{\drawboxvoidbar}}
\def\luangle{\text{\mbox{$\updownline\hskip-6.13pt\raisebox{4.37pt}{$\leftrightline$}$}}}
\def\rdangle{\text{\mbox{$\raisebox{-4.37pt}{$\leftrightline$}\hskip-6.13pt\updownline$}}}
\def\dbluangle {\raisebox{\raising}{\drawdbluangle}}


\def\leftpull {\boxtimes\hskip-.37em\Box}
\def\leftpush {\boxplus\hskip-.37em\Box}
\def\rightpull{\Box\hskip-.37em\boxtimes}
\def\rightpush{\Box\hskip-.37em\boxplus}
\def\nboxes   {\square\hskip-.16em\square\cdots\square}

%===================[ homotopy orthogonality symbol ]
\newcommand{\htth}{{\scriptsize $\triangle$}\textsc{tth}\@\xspace}
\newcommand{\hfs}{{\scriptsize $\triangle$}\textsc{fs}\@\xspace}
\newcommand{\phfs}{{\scriptsize $\triangle$}\textsc{pfs}\@\xspace}

\def\hdisplayed{\mathbin{\ooalign{\hfil\raisebox{.255em}{$\shortmid$}\hfil\cr\raisebox{0em}{\scalebox{1}[.6]{$\approx$}}\cr}}}
\def\hnormaled{\hdisplayed}
\def\hscripted{\ooalign{\hfil\raisebox{.18em}{$\scriptstyle\shortmid$}\hfil\cr\raisebox{0em}{\scalebox{1}[.6]{$\scriptstyle\approx$}}\cr}}
\def\hscriptscripted{\ooalign{\hfil\raisebox{.12em}{$\scriptscriptstyle\shortmid$}\hfil\cr\raisebox{0em}{\scalebox{1}[.6]{$\scriptscriptstyle\approx$}}\cr}}
%
%
%
\def\cdisplayed{\mathbin{\ooalign{\hfil\raisebox{.225em}{$\shortmid$}\hfil\cr\raisebox{-.12em}{\scalebox{1}[1]{$=$}}\cr}}}
\def\cnormaled{\cdisplayed}
\def\cscripted{\ooalign{\hfil\raisebox{.14em}{$\scriptstyle\shortmid$}\hfil\cr\raisebox{-.12em}{\scalebox{1}[1]{$\scriptstyle =$}}\cr}}
\def\cscriptscripted{\ooalign{\hfil\raisebox{.05em}{$\scriptscriptstyle\shortmid$}\hfil\cr\raisebox{-.12em}{\scalebox{1}[1]{$\scriptscriptstyle =$}}\cr}}
%
\def\horth{\mathchoice{\hdisplayed}{\hnormaled}{\hscripted}{\hscriptscripted}}%
\def\corth{\mathchoice{\cdisplayed}{\cnormaled}{\cscripted}{\cscriptscripted}}%

\newcommand{\var}[2]{\left[ \begin{smallmatrix} #1 \\ \downarrow \\ #2 \end{smallmatrix}\right]}
\newcommand{\smat}[1]{\left[ \begin{smallmatrix} #1 \end{smallmatrix}\right]}

% famola sporchissima

\def\lhorth#1{\leftidx{^{\horth}}{#1}{}}
\def\prescript#1#2#3{\leftidx{^{#1}}{_{#2}}{#3}}

\def\orth{\boxslash}


%===================[ homotopy orthogonality symbol ]
\def\mho{\rotatebox[origin=c]{180}{$\Omega$}}

\DeclareMathOperator{\id}{id}
\DeclareMathOperator{\psh}{PSh}
\DeclareMathOperator{\holim}{holim}
\DeclareMathOperator{\hocolim}{hocolim}

\def\Nat{\textsf{Nat}}
\def\PsNat{\textsf{PsNat}}
\def\LNat{\textsf{LNat}}
\usepackage{CJKutf8}

\newcommand{\yon}{\text{\begin{CJK}{UTF8}{min}よ\end{CJK}}\!}

\DeclareMathAlphabet\EuScript{U}{eus}{m}{n}
\SetMathAlphabet\EuScript{bold}{U}{eus}{b}{n}
\newcommand{\CMcal}{}

\def\cate#1{\textbf{#1}}
	\def\S  {\cate{S}}
	\def\C  {\mathcal{C}}
	\def\iC{\EuScript{C}}
	\def\sD {\cate{D}}
	\def\T  {\cate{T}}
	\def\B {\mathcal{B}}
	\def\cD {\mathcal{D}}
	\def\fincat{\cate{fdCat}}
	\def\PDer{\cate{PDer}}
	\def\Der{\cate{Der}}

\def\Fun{\mathrm{Fun}}
\def\bDelta{\boldsymbol{\Delta}}

	\def\F {\mathfrak{F}}
	\def\fF{\F}
\def\fs#1{\mathbb{#1}}
	\def\D {\fs{D}}
	\def\X {\fs{X}}
	\def\fS{\fs{S}}

\def\class#1{\mathcal{#1}}
	\def\M{\class{M}}
	\def\E{\class{E}}
	\def\A{\class{A}}
  	\def\K{\class{K}}
	\def\V{\class{V}}

\newcommand{\xto}[1]{\xrightarrow{#1}}
\newcommand{\xot}[1]{\xleftarrow{#1}}
\newcommand{\To}{\Rightarrow}

\def\Func{\mathrm{Func}}
\def\Ob{\mathrm{Ob}}
\def\Dia{\cate{Dia}}
\def\dia{\text{dia}}
\def\opp{\text{op}}
\def\co{\text{co}}
\def\coop{\text{coop}}
\def\ho{\textsf{Ho}}
\def\pf{\textsc{pf}}
\def\ts{\textsc{ts}}
\def\dfs{\textsc{dfs}\@\xspace}
\def\pt{\text{pt}}
\def\tre{\textbf{3}}
\def\due{\textbf{2}}
\def\uno{\textbf{1}}

\def\numbers#1{\mathbb{#1}}
	\def\Z{\numbers{Z}}
	\def\N{\numbers{N}}

\def\derivator#1{\mathbb{#1}}
\def\derC{\derivator{C}}
\def\derA{\derivator{A}}
\def\derB{\derivator{B}}
\def\P   {\derivator{P}}

\newcommand{\Nearrow}{\rotatebox[origin=c]{45}{$\Rightarrow$}}
\newcommand{\Nwarrow}{\rotatebox[origin=c]{135}{$\Rightarrow$}}
\newcommand{\Searrow}{\rotatebox[origin=c]{-45}{$\Rightarrow$}}
\newcommand{\Swarrow}{\rotatebox[origin=c]{225}{$\Rightarrow$}}
\newcommand{\Sarrow}{\rotatebox[origin=c]{-90}{$\Rightarrow$}}
\newcommand{\Narrow}{\rotatebox[origin=c]{90}{$\Rightarrow$}}

\def\restriction{|}
\def\TT{\textsf{T}}

\usepackage{bm}
\def\circledast{\protect{\tiny \oast}}%\protect{\boldsymbol{*}}}
\def\circledbang{\protect{\rotatebox[origin=c]{90}{\tiny $\ominus$}}}%\protect{\boldsymbol{!}}}
\def\ee{e}
\def\mm{m}
\def\cpfs{\textsc{dpfs}\@\xspace}
\def\dpfs{\textsc{dpfs}\@\xspace}

\usepackage{turnstile}
	\newcommand{\adjunct}[2]{\nsststile{#2}{#1}}

\newcommand{\deduction}[4]{
	\begin{array}{c}
		#1 \to #2 \\ \hline
		#3 \to #4
	\end{array}
}
\def\Lan{\text{Lan}}
\def\Ran{\text{Ran}}
\def\Lift{\text{Lift}}
\def\Rift{\text{Rift}}
\def\K{\mathcal{K}}
\def\y{\textbf{y}}

\def\rift{\text{rift}}
\def\leeft{\text{lift}} % `lift is already something!
\def\lan{\text{lan}}
\def\ran{\text{ran}}
\def\Rift{\text{Rift}}
\def\Lift{\text{Lift}}
\def\Ran{\text{Ran}} 
\def\Lan{\text{Lan}}
\def\RIFT{\textsc{rift}}
\def\LIFT{\textsc{lift}}
\def\RAN{\textsc{ran}}
\def\LAN{\textsc{lan}}

\def\adm{\text{adm}}
\def\ct{\cate{cat}}
\def\Cat{\cate{Cat}}
\def\CAT{\cate{CAT}}
\def\set{\cate{set}}
\def\Set{\cate{Set}}
\def\SET{\cate{SET}}

\def\smlerp{\text{\wasylozenge}}

%===================[ layout ]
\setlist[1]{itemsep=0pt}

\renewcommand{\textbf}[1]{\text{\fontseries{b}\selectfont{\upshape #1}}}
\def\lastcompiled{\textcolor{red}{\noindent\bf Last compiled: \today \hfill\currenttime}}


\newcommand{\arXivPreprint}[1]{arXiv preprint \href{http://arxiv.org/abs/#1}{arXiv:#1}}

\newtheoremstyle{reference}%
   {}
   {}
   {}                      % Font del testo
   {}                      % Rientro margini
   {\bfseries}             % Font del titolo dell'ambiente
   {:}                     % Punteggiatura dopo "Teorema"\"Definizione"
   {.2em}                  % Spazio tra titolo e testo.
   {\thmname{#1}           % #1 : Definizione\Teorema\ecc
    \thmnumber{#2}         % #2 : Contatore
    \thmnote{{\sc [#3]}}}  % #3 : Testo tra "[" e "]"

\theoremstyle{reference}
  \newtheorem{theorem}{Theorem}[section]
  \newtheorem{conjec}[theorem]{Conjecture}
  \newtheorem{corollary}[theorem]{Corollary}
  \newtheorem{counterex}[theorem]{Counterexample}
  \newtheorem{definition}[theorem]{Definition}
  \newtheorem{example}[theorem]{Example}
  \newtheorem{exercise}[theorem]{Exercise}
  \newtheorem{lemma}[theorem]{Lemma}
  \newtheorem{notat}[theorem]{Notation}
  \newtheorem{proposition}[theorem]{Proposition}
  \newtheorem{question}[theorem]{Question}
  \newtheorem{remark}[theorem]{Remark}
  \newtheorem{scholium}[theorem]{Scholium}
  \newtheorem{setting}[theorem]{Setting}
  \newtheorem{conjecture}[theorem]{Conjecture}
  % starred
  \newtheorem*{conjec*}{Conjecture}
  \newtheorem*{corollary*}{Corollary}
  \newtheorem*{counterex*}{Counterexample}
  \newtheorem*{definition*}{Definition}
  \newtheorem*{example*}{Example}
  \newtheorem*{exercise*}{Exercise}
  \newtheorem*{lemma*}{Lemma}
  \newtheorem*{notat*}{Notation}
  \newtheorem*{proposition*}{Proposition}
  \newtheorem*{question*}{Question}
  \newtheorem*{remark*}{Remark}
  \newtheorem*{scholium*}{Scholium}
  \newtheorem*{setting*}{Setting}
  \newtheorem*{theorem*}{Theorem}

% ########################################
% ALCUNE COSE VANNO SEMPRE IN FONDO

\hypersetup{%
  pdftoolbar=   true,
  pdfmenubar=   true,
  pdffitwindow= true,
  pdftitle=     {Presentable derivators},
  pdfauthor=    {Loregian - Rosicky - Virili},
  colorlinks=   true,
  linkcolor=    black,
  citecolor=    blue!40!black}


\providecommand{\refbf}[1]{\textbf{\ref{#1}}}
% stile del comando \cite
\makeatletter
  \def\@cite#1#2{[\textbf{#1}\if@tempswa , #2\fi]}
  \def\@biblabel#1{[\textsf{#1}]}
\makeatother


\providecommand{\abbrv}[1]{#1.\@\xspace}
  \providecommand{\ie}{\abbrv{i.e}}
  \providecommand{\etc}{\abbrv{etc}}
  \providecommand{\prof}{\abbrv{prof}}
  \providecommand{\viz}{\abbrv{viz}}
  \providecommand{\eg}{\abbrv{e.g}}
  \providecommand{\achap}{\abbrv{Ch}}
  \providecommand{\adef}{\abbrv{Def}}
  \providecommand{\acor}{\abbrv{Cor}}
  \providecommand{\aprop}{\abbrv{Prop}}
  \providecommand{\athm}{\abbrv{Thm}}


\newlength{\seplen}
\setlength{\seplen}{5pt}
%
\newlength{\sepwid}
\setlength{\sepwid}{.4pt}
%
\def\firstblank{\,\rule{\seplen}{\sepwid}\,}
\def\secondblank{\firstblank\llap{\raisebox{2pt}{\firstblank}}}

\setcounter{tocdepth}{1}

\newcommand{\mailto}[1]{\href{mailto:#1}{\sf #1}}
\newcommand{\fcomment}[1]{\todo[inline,caption={}]{\textbf{F says} : #1}}

% REMOVE THIS TO GET BACK AMSART HEADERS ========
\pagestyle{fancy}
\chead{\texttt{\tiny \color{red}TeX 3.14159265 (TeX Live 2016)
 : \today \hfill\currenttime}}
\lhead{}
\rhead{}
\cfoot{}%{\texttt{\tiny \color{red}TeX 3.14159265 (TeX Live 2016)
 : \today \hfill\currenttime}}
\lfoot[\thepage]{}
\rfoot[]{\thepage}

\hyphenation{or-tho-go-na-li-ty qua-si-ca-te-go-ry qua-si-ca-te-go-ries de-ri-va-tor Gro-then-dieck through-out co-mo-noid ge-ne-ra-li-ty in-jec-ti-vi-ty}

\allowdisplaybreaks
%\pretolerance=1500
\renewcommand{\textbf}[1]{\text{\fontseries{b}\selectfont{\upshape #1}}}
\usepackage[all,cmtip,2cell]{xy}\UseAllTwocells

\def\hfp{{\sf hfp}}
\usepackage{tikz-cd}
\title{On accessible prederivators}
\author{}
\begin{document}
\maketitle
\paragraph{\bf Conventions and notation.} 
We shall call \emph{$\sf Term$-cocomplete} (\abbrv{resp} \emph{$\sf Term$-complete}) a prederivator that admits left (\abbrv{resp} right) Kan extensions $t_{A,!}$ along all terminal functors $t_A : A \to e$.  As it is customary, we denote $\hocolim_A X$ the object $t_{A,!}X \in\D(e)$ obtained from a `coherent diagram' $X\in\D(A)$. Each such $X$ has an associated underlying diagram $\dia_AX \in \D(e)^A$, defined as $a\mapsto a^*X$. For every object $Y\in \D(e)$ we consider the composition
\[
\xymatrix@C=2cm@R=0cm{A \ar[r]^{\dia_AX}& \D(e)\ar[r]^{\D(e)(Y,\firstblank)} & \Set\\
a \ar@{|->}[rr]&& \D(e)(Y, a^*X)}
\]

If $\lambda$ is a regular cardinal we shall denote $\lambda\textsf{-Filt}$ the class of filtered categories, and ${\sf Cat}_{<\lambda}$ the class of categories with less than $\lambda$ objects. $\sf fFilt$ and $\sf fCat$ are shorthand for $\omega\textsf{-Filt}$ and ${\sf Cat}_{<\omega}$ respectively. We will informally call a subcategory of $\Cat$ a \emph{doctrine}.
\section{Accessible prederivators}
\begin{definition}[homotopy presentable object]
Let $\D$ be a $\sf Term$-cocomplete prederivator, and $C\in\D(e)$ an object in the base. We say that $C$ is \emph{homotopy finitely presentable} (\emph{hfp} for short) if for every category $I\in\textsf{fFilt}$ and every $X\in\D(I)$ there is an isomorphism
\[\textstyle
\varinjlim_I \D(e)(C, i^*X) \cong \D(e)(C, \hocolim_I X)
\]
induced by the canonical diagram $\dia_I X$ of $X$.
\end{definition}
\begin{remark}
Of course this definition can be extended to the members of an arbitrary doctrine $\sf D$ different from $\sf fFilt$: in this case we speak of \emph{homotopy ${\sf D}$-presentable objects} and accordingly change the terminology elsewhere; it is obvious what a \emph{homotopy $\lambda$-presentable object} is, according to this definition.
\end{remark}
\begin{remark}[relativization]
The following conditions are equivalent:
\begin{itemize}
	\item The object $C$ is hfp for the shifted derivator $\D^J$;
	\item For every category $I\in\sf fFilt$ and every $X\in\D(I)$ there is an isomorphism
	\[\textstyle
	\D(J)(C, p_!X)\cong \varinjlim_I \D(J)(C, (i\times 1)^* X)
	\]
	where $i\times 1\colon J\to I\times J\colon j\mapsto (i,j)$ and $p \colon I\times J \to J$ is the projection. This is evidently a condition on the partial underlying diagram functor $\dia_{I,J}X$ of \cite{groth2013derivators}
\end{itemize}
\end{remark}
Thanks to the above remark, the following definition makes sense.
\begin{definition}[the sub-prederivator $\hfp(\D)$]
We call \emph{hfp objects at $J$} the hfp objects of the shifted derivator $\D^J$; the hfp objects at $J$ of $\D$ are canonically identified with a subcategory $\hfp(\D(J))$ of $\D(J)$; the inclusions $\hfp(\D(J)) \hookrightarrow \D(J)$ assemble into a morphism of prederivators $\iota_\hfp : \hfp(\D) \hookrightarrow \D$; we call the sub-prederivator $\hfp(\D)$ the sub-prederivator of \emph{hfp objects} of $\D$.
\end{definition}
\begin{definition}[accessible prederivator]
A prederivator $\D$ is called \emph{accessible} if it satisfies the following conditions:
\begin{itemize}
	\item Its prederivator of hfp objects is small;
	\item The left Kan extension of $\iota_\hfp$ along itself is isomorphic to $\id_\D$.
\end{itemize}
\end{definition}
\begin{remark}
Let $\D_\iC$ be the represented derivator of a category $\iC$, that sends $J\mapsto \iC^J$; then an object $C$ is hfp at $J$ if and only if it is finitely presentable in $\iC^J$. In particular $C$ is hfp if and only if it is finitely presentable in $\iC$
\end{remark}
\section{A few questions}
\begin{itemize}
	\item Let $\M$ be a combinatorial model category and $\D_\M$ be its associated derivator. Let $C$ be an object which fibrant and cofibrant and compact in $\M$. Then $C$ is hfp in $\D_\M(e) = \ho(\M)$.
	\item Let $\iC$ be an accessible $\infty$-category, $C\in\iC$ an object and $\D_\iC$ its associated prederivator; then $C$ is hfp in $\D_\iC(e)=\ho(\iC)$.
	\item hfp objects are closed under finite homotopy colimits, \ie if $J$ is a finite category, $X\in\D(J)$ is a coherent diagram of shape $J$, and each $j^*X\in\D(e)$ is an hfp object, then $\hocolim_J X \in\D(e)$ is an hfp object.
\end{itemize}
\cleardoublepage

\section{A different turn}
Let $\K$ be a 2-category; we say that $A\in \K$ has a \emph{strong generator} if there exists a fully faithful 1-cell $j : G \to A$ such that $\Lan_jj = 1$.

Let $\K$ be a 2-category with a Yoneda structure, and $\lambda$ a regular cardinal; an object $A\in\K$ is $\lambda$-accessible if 
\begin{itemize}
	\item it admits a dense generator $j : G \to A$;
	\item the 2-cell
\[
\xymatrix{
	A \ar@{}[dr]|(.3)\Swarrow\ar[r]^y \ar[d]_y & PA \\
	PA \ar[ur]_{T_{\lambda\text{-Filt}}} & 
}
\]
corresponding to the unit $\eta$ of the monad becomes invertible when pasted with $j$.
\end{itemize}
This means that every object of $G$ is ``$\lambda$-presentable'', since TFAE (in $\Cat$):
\begin{itemize}
	\item $a\in A$ is a $\lambda$-presentable object;
	\item $a\mapsto \hom(-,a)$ commutes with $\lambda$-filtered colimits;
	\item $T_{\lambda\text{-Filt}}(\hom(-,a))\cong \hom(-,a)$ where $T_{\lambda\text{-Filt}}$ is the monad of $\lambda$-filtered functors that reflects $PA=[A^\opp,\Set]$ into $[A^\opp,Set]^{\lambda\text{-Filt}}$.
\end{itemize}
\begin{remark}
An object $A\in\K$ is \emph{locally $\lambda$-presentable} with respect to the Yoneda structure if it admits a generator $J$ and $A(J,1)$ has a fully faithful left adjoint (read ``$A$ is a cocomplete localization of $PG$'').

The main theorem in this respect is the equation $LP = Acc + coc$. This is what we shall prove first. The bare idea is that in the diagram
\[
\xymatrix@R=2cm{
&& PA \ar[drr]&& \\ 
G \ar[rr]^j\ar[dr]&& A
\ar@{}[urr]|(.3){\Nwarrow\alpha}
\ar@{}[drr]|(.4)\Nearrow
\ar@{}[dll]|\Sarrow
\ar@{}[d]|{\Sarrow\;\epsilon}
\ar[dr]\ar[u]\ar[rr]&& PA\ar[dl] \\
& PG \ar[ur]\ar@{=}[rr]&& PG &
}
\]
we have $\alpha * j$ iso iff $\varepsilon$ iso.

A few remarks on accessibility and admissibility, as the two notions seem related (both are ``smallness'' requests):
\begin{itemize}
	\item An uniformly accessible category is an accessible category where the accessibility degree of objects is bounded by som $\mu$; a uniformly accessible category is small (the precise equation is $UnAcc = small + Cauchy complete$). 
	\item Copy the definition in a Yoneda structure $\mathfrak Y$: is it possible to show that the $\mathfrak Y$-accessibility notion implies that a uniformly $\mathfrak Y$-accessible object is admissible and ``Cauchy complete'' (whatever it means in $\mathfrak Y$)?
\end{itemize}

\end{remark}
This definition is relative to the Yoneda structure as it employs the Yoneda arrows $y_G : G \to PG$, $y_A : A \to PA$, etc.
\bibliographystyle{amsalpha}{}
\bibliography{allofthem}
\end{document}