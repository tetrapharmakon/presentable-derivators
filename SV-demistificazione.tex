\documentclass[a4paper,10pt]{amsart}

\usepackage[OT2,T1]{fontenc}
\usepackage[utf8]{inputenc}
\usepackage[russian,english]{babel}
\usepackage{
   amsmath
  ,amsfonts
  ,amssymb
  ,amsthm
  ,leftidx
  ,cite
  ,datetime
  ,enumitem
  ,stmaryrd
  ,todonotes
  ,xspace
  ,xparse
  ,hyphenat
  ,epigraph
  ,kodi
  ,faktor
  ,mathtools
  ,fancyhdr
  ,blindtext
  ,tikz
  ,tikz-cd
  ,wasysym
  ,marginnote
  % ,fdsymbol
}
\usepackage[all,cmtip,2cell]{xy}\UseAllTwocells

\usepackage[cal=pxtx]{mathalfa}
\usepackage{hyperref}
\usepackage{cjhebrew}

%===================[ squares and stuph ]
\usepackage{tikz,tikz-cd}
\usetikzlibrary{arrows}
\usetikzlibrary{babel}

\tikzset{%
implies/.style={double,double equal sign distance,-implies},
shorten <>/.style={shorten >=#1,shorten <=#1}}

\newlength{\spacing}
\setlength{\spacing}{0pt}
%
\def\scaling{.2}
%
\newlength{\raising}
\setlength{\raising}{.5pt}

\def\drawboxvoid{
\begin{tikzpicture}[scale=\scaling]
\draw (0,0) rectangle (1,1);%
\end{tikzpicture}}
%%%%%
\def\drawluangle{\begin{tikzpicture}[scale=\scaling]
\draw (0,0) -- (0,1) -- (1,1);%
\end{tikzpicture}}
%%%%%
\def\drawdbluangle{\begin{tikzpicture}[scale=\scaling]
\draw (0,0) -- (0,1) -- (1,1);%
\end{tikzpicture}}
%%%%%
\def\drawrdangle{\begin{tikzpicture}[scale=\scaling]
\draw (0,0) -- (1,0) -- (1,1);%
\end{tikzpicture}}
%%%%%
\def\drawtwoboxes{\begin{tikzpicture}[scale=\scaling]
\draw (0,0) -- (1,0) -- (1,1) -- (0,1) -- cycle;%
\draw[xshift=1cm] (0,0) -- (1,0) -- (1,1) -- (0,1) -- cycle;%
\end{tikzpicture}}
%%%%%
\def\drawldangle{\begin{tikzpicture}[scale=\scaling]
\draw (1,0) -- (0,0) -- (0,1);%
\end{tikzpicture}}
%%%%%
\def\drawboxvoidbar{\begin{tikzpicture}[scale=\scaling]
\draw (1,1) -- (1,0) -- (0,0) -- (0,1) -- (2,1);%
\end{tikzpicture}}
%%%%%
\def\boxvoid   {\raisebox{\raising}{\drawboxvoid}}
\def\twoboxes  {\raisebox{\raising}{\drawtwoboxes}}
\def\ldangle   {\raisebox{\raising}{\drawldangle}}
\def\boxvoidbar{\raisebox{\raising}{\drawboxvoidbar}}
\def\luangle{\text{\mbox{$\updownline\hskip-6.13pt\raisebox{4.37pt}{$\leftrightline$}$}}}
\def\rdangle{\text{\mbox{$\raisebox{-4.37pt}{$\leftrightline$}\hskip-6.13pt\updownline$}}}
\def\dbluangle {\raisebox{\raising}{\drawdbluangle}}


\def\leftpull {\boxtimes\hskip-.37em\Box}
\def\leftpush {\boxplus\hskip-.37em\Box}
\def\rightpull{\Box\hskip-.37em\boxtimes}
\def\rightpush{\Box\hskip-.37em\boxplus}
\def\nboxes   {\square\hskip-.16em\square\cdots\square}

%===================[ homotopy orthogonality symbol ]
\newcommand{\htth}{{\scriptsize $\triangle$}\textsc{tth}\@\xspace}
\newcommand{\hfs}{{\scriptsize $\triangle$}\textsc{fs}\@\xspace}
\newcommand{\phfs}{{\scriptsize $\triangle$}\textsc{pfs}\@\xspace}

\def\hdisplayed{\mathbin{\ooalign{\hfil\raisebox{.255em}{$\shortmid$}\hfil\cr\raisebox{0em}{\scalebox{1}[.6]{$\approx$}}\cr}}}
\def\hnormaled{\hdisplayed}
\def\hscripted{\ooalign{\hfil\raisebox{.18em}{$\scriptstyle\shortmid$}\hfil\cr\raisebox{0em}{\scalebox{1}[.6]{$\scriptstyle\approx$}}\cr}}
\def\hscriptscripted{\ooalign{\hfil\raisebox{.12em}{$\scriptscriptstyle\shortmid$}\hfil\cr\raisebox{0em}{\scalebox{1}[.6]{$\scriptscriptstyle\approx$}}\cr}}
%
%
%
\def\cdisplayed{\mathbin{\ooalign{\hfil\raisebox{.225em}{$\shortmid$}\hfil\cr\raisebox{-.12em}{\scalebox{1}[1]{$=$}}\cr}}}
\def\cnormaled{\cdisplayed}
\def\cscripted{\ooalign{\hfil\raisebox{.14em}{$\scriptstyle\shortmid$}\hfil\cr\raisebox{-.12em}{\scalebox{1}[1]{$\scriptstyle =$}}\cr}}
\def\cscriptscripted{\ooalign{\hfil\raisebox{.05em}{$\scriptscriptstyle\shortmid$}\hfil\cr\raisebox{-.12em}{\scalebox{1}[1]{$\scriptscriptstyle =$}}\cr}}
%
\def\horth{\mathchoice{\hdisplayed}{\hnormaled}{\hscripted}{\hscriptscripted}}%
\def\corth{\mathchoice{\cdisplayed}{\cnormaled}{\cscripted}{\cscriptscripted}}%

\newcommand{\var}[2]{\left[ \begin{smallmatrix} #1 \\ \downarrow \\ #2 \end{smallmatrix}\right]}
\newcommand{\smat}[1]{\left[ \begin{smallmatrix} #1 \end{smallmatrix}\right]}

% famola sporchissima

\def\lhorth#1{\leftidx{^{\horth}}{#1}{}}
\def\prescript#1#2#3{\leftidx{^{#1}}{_{#2}}{#3}}

\def\orth{\boxslash}


%===================[ homotopy orthogonality symbol ]
\def\mho{\rotatebox[origin=c]{180}{$\Omega$}}

\DeclareMathOperator{\id}{id}
\DeclareMathOperator{\psh}{PSh}
\DeclareMathOperator{\holim}{holim}
\DeclareMathOperator{\hocolim}{hocolim}

\def\Nat{\textsf{Nat}}
\def\PsNat{\textsf{PsNat}}
\def\LNat{\textsf{LNat}}
\usepackage{CJKutf8}

\newcommand{\yon}{\text{\begin{CJK}{UTF8}{min}よ\end{CJK}}\!}

\DeclareMathAlphabet\EuScript{U}{eus}{m}{n}
\SetMathAlphabet\EuScript{bold}{U}{eus}{b}{n}
\newcommand{\CMcal}{}

\def\cate#1{\textbf{#1}}
	\def\S  {\cate{S}}
	\def\C  {\mathcal{C}}
	\def\iC{\EuScript{C}}
	\def\sD {\cate{D}}
	\def\T  {\cate{T}}
	\def\B {\mathcal{B}}
	\def\cD {\mathcal{D}}
	\def\fincat{\cate{fdCat}}
	\def\PDer{\cate{PDer}}
	\def\Der{\cate{Der}}

\def\Fun{\mathrm{Fun}}
\def\bDelta{\boldsymbol{\Delta}}

	\def\F {\mathfrak{F}}
	\def\fF{\F}
\def\fs#1{\mathbb{#1}}
	\def\D {\fs{D}}
	\def\X {\fs{X}}
	\def\fS{\fs{S}}

\def\class#1{\mathcal{#1}}
	\def\M{\class{M}}
	\def\E{\class{E}}
	\def\A{\class{A}}
  	\def\K{\class{K}}
	\def\V{\class{V}}

\newcommand{\xto}[1]{\xrightarrow{#1}}
\newcommand{\xot}[1]{\xleftarrow{#1}}
\newcommand{\To}{\Rightarrow}

\def\Func{\mathrm{Func}}
\def\Ob{\mathrm{Ob}}
\def\Dia{\cate{Dia}}
\def\dia{\text{dia}}
\def\opp{\text{op}}
\def\co{\text{co}}
\def\coop{\text{coop}}
\def\ho{\textsf{Ho}}
\def\pf{\textsc{pf}}
\def\ts{\textsc{ts}}
\def\dfs{\textsc{dfs}\@\xspace}
\def\pt{\text{pt}}
\def\tre{\textbf{3}}
\def\due{\textbf{2}}
\def\uno{\textbf{1}}

\def\numbers#1{\mathbb{#1}}
	\def\Z{\numbers{Z}}
	\def\N{\numbers{N}}

\def\derivator#1{\mathbb{#1}}
\def\derC{\derivator{C}}
\def\derA{\derivator{A}}
\def\derB{\derivator{B}}
\def\P   {\derivator{P}}

\newcommand{\Nearrow}{\rotatebox[origin=c]{45}{$\Rightarrow$}}
\newcommand{\Nwarrow}{\rotatebox[origin=c]{135}{$\Rightarrow$}}
\newcommand{\Searrow}{\rotatebox[origin=c]{-45}{$\Rightarrow$}}
\newcommand{\Swarrow}{\rotatebox[origin=c]{225}{$\Rightarrow$}}
\newcommand{\Sarrow}{\rotatebox[origin=c]{-90}{$\Rightarrow$}}
\newcommand{\Narrow}{\rotatebox[origin=c]{90}{$\Rightarrow$}}

\def\restriction{|}
\def\TT{\textsf{T}}

\usepackage{bm}
\def\circledast{\protect{\tiny \oast}}%\protect{\boldsymbol{*}}}
\def\circledbang{\protect{\rotatebox[origin=c]{90}{\tiny $\ominus$}}}%\protect{\boldsymbol{!}}}
\def\ee{e}
\def\mm{m}
\def\cpfs{\textsc{dpfs}\@\xspace}
\def\dpfs{\textsc{dpfs}\@\xspace}

\usepackage{turnstile}
	\newcommand{\adjunct}[2]{\nsststile{#2}{#1}}

\newcommand{\deduction}[4]{
	\begin{array}{c}
		#1 \to #2 \\ \hline
		#3 \to #4
	\end{array}
}
\def\Lan{\text{Lan}}
\def\Ran{\text{Ran}}
\def\Lift{\text{Lift}}
\def\Rift{\text{Rift}}
\def\K{\mathcal{K}}
\def\y{\textbf{y}}

\def\rift{\text{rift}}
\def\leeft{\text{lift}} % `lift is already something!
\def\lan{\text{lan}}
\def\ran{\text{ran}}
\def\Rift{\text{Rift}}
\def\Lift{\text{Lift}}
\def\Ran{\text{Ran}} 
\def\Lan{\text{Lan}}
\def\RIFT{\textsc{rift}}
\def\LIFT{\textsc{lift}}
\def\RAN{\textsc{ran}}
\def\LAN{\textsc{lan}}

\def\adm{\text{adm}}
\def\ct{\cate{cat}}
\def\Cat{\cate{Cat}}
\def\CAT{\cate{CAT}}
\def\set{\cate{set}}
\def\Set{\cate{Set}}
\def\SET{\cate{SET}}

\def\smlerp{\text{\wasylozenge}}

%===================[ layout ]
\setlist[1]{itemsep=0pt}

\renewcommand{\textbf}[1]{\text{\fontseries{b}\selectfont{\upshape #1}}}
\def\lastcompiled{\textcolor{red}{\noindent\bf Last compiled: \today \hfill\currenttime}}


\newcommand{\arXivPreprint}[1]{arXiv preprint \href{http://arxiv.org/abs/#1}{arXiv:#1}}

\newtheoremstyle{reference}%
   {}
   {}
   {}                      % Font del testo
   {}                      % Rientro margini
   {\bfseries}             % Font del titolo dell'ambiente
   {:}                     % Punteggiatura dopo "Teorema"\"Definizione"
   {.2em}                  % Spazio tra titolo e testo.
   {\thmname{#1}           % #1 : Definizione\Teorema\ecc
    \thmnumber{#2}         % #2 : Contatore
    \thmnote{{\sc [#3]}}}  % #3 : Testo tra "[" e "]"

\theoremstyle{reference}
  \newtheorem{theorem}{Theorem}[section]
  \newtheorem{conjec}[theorem]{Conjecture}
  \newtheorem{corollary}[theorem]{Corollary}
  \newtheorem{counterex}[theorem]{Counterexample}
  \newtheorem{definition}[theorem]{Definition}
  \newtheorem{example}[theorem]{Example}
  \newtheorem{exercise}[theorem]{Exercise}
  \newtheorem{lemma}[theorem]{Lemma}
  \newtheorem{notat}[theorem]{Notation}
  \newtheorem{proposition}[theorem]{Proposition}
  \newtheorem{question}[theorem]{Question}
  \newtheorem{remark}[theorem]{Remark}
  \newtheorem{scholium}[theorem]{Scholium}
  \newtheorem{setting}[theorem]{Setting}
  \newtheorem{conjecture}[theorem]{Conjecture}
  % starred
  \newtheorem*{conjec*}{Conjecture}
  \newtheorem*{corollary*}{Corollary}
  \newtheorem*{counterex*}{Counterexample}
  \newtheorem*{definition*}{Definition}
  \newtheorem*{example*}{Example}
  \newtheorem*{exercise*}{Exercise}
  \newtheorem*{lemma*}{Lemma}
  \newtheorem*{notat*}{Notation}
  \newtheorem*{proposition*}{Proposition}
  \newtheorem*{question*}{Question}
  \newtheorem*{remark*}{Remark}
  \newtheorem*{scholium*}{Scholium}
  \newtheorem*{setting*}{Setting}
  \newtheorem*{theorem*}{Theorem}

% ########################################
% ALCUNE COSE VANNO SEMPRE IN FONDO

\hypersetup{%
  pdftoolbar=   true,
  pdfmenubar=   true,
  pdffitwindow= true,
  pdftitle=     {Presentable derivators},
  pdfauthor=    {Loregian - Rosicky - Virili},
  colorlinks=   true,
  linkcolor=    black,
  citecolor=    blue!40!black}


\providecommand{\refbf}[1]{\textbf{\ref{#1}}}
% stile del comando \cite
\makeatletter
  \def\@cite#1#2{[\textbf{#1}\if@tempswa , #2\fi]}
  \def\@biblabel#1{[\textsf{#1}]}
\makeatother


\providecommand{\abbrv}[1]{#1.\@\xspace}
  \providecommand{\ie}{\abbrv{i.e}}
  \providecommand{\etc}{\abbrv{etc}}
  \providecommand{\prof}{\abbrv{prof}}
  \providecommand{\viz}{\abbrv{viz}}
  \providecommand{\eg}{\abbrv{e.g}}
  \providecommand{\achap}{\abbrv{Ch}}
  \providecommand{\adef}{\abbrv{Def}}
  \providecommand{\acor}{\abbrv{Cor}}
  \providecommand{\aprop}{\abbrv{Prop}}
  \providecommand{\athm}{\abbrv{Thm}}


\newlength{\seplen}
\setlength{\seplen}{5pt}
%
\newlength{\sepwid}
\setlength{\sepwid}{.4pt}
%
\def\firstblank{\,\rule{\seplen}{\sepwid}\,}
\def\secondblank{\firstblank\llap{\raisebox{2pt}{\firstblank}}}

\setcounter{tocdepth}{1}

\newcommand{\mailto}[1]{\href{mailto:#1}{\sf #1}}
\newcommand{\fcomment}[1]{\todo[inline,caption={}]{\textbf{F says} : #1}}

% REMOVE THIS TO GET BACK AMSART HEADERS ========
\pagestyle{fancy}
\chead{\texttt{\tiny \color{red}TeX 3.14159265 (TeX Live 2016)
 : \today \hfill\currenttime}}
\lhead{}
\rhead{}
\cfoot{}%{\texttt{\tiny \color{red}TeX 3.14159265 (TeX Live 2016)
 : \today \hfill\currenttime}}
\lfoot[\thepage]{}
\rfoot[]{\thepage}

\hyphenation{or-tho-go-na-li-ty qua-si-ca-te-go-ry qua-si-ca-te-go-ries de-ri-va-tor Gro-then-dieck through-out co-mo-noid ge-ne-ra-li-ty in-jec-ti-vi-ty}

\allowdisplaybreaks
%\pretolerance=1500
\title{Conspectus, debunked}
\author{Fouche}
\def\eent{\!\!\int\!\!}
\def\H{\textbf{H}}
\theoremstyle{definition}
\newtheorem{prg}{}[section]
\begin{document}
\maketitle
\begin{abstract}
Questa nota serve (a entrambi) per capire meglio cosa c'è dentro \cite{street1981conspectus}
\end{abstract}
\section{\cite{street1981conspectus}, §1}
\begin{prg}
C'è un funtore
\[
\xymatrix@R=0cm{
	[\C^\opp, \Cat] \ar[r]^\eent & [\C^\co,\cate{Hom}]\\
	\text{functors} & \text{functors}\\
	{\begin{rcases*}
	\text{\tiny pseudo}\\[-5pt]
	\text{\tiny co/lax}\\[-5pt]
	\text{\tiny strict}
	\end{rcases*}
	\text{transf.}} & 
	{\begin{rcases*}
	\text{\tiny pseudo}\\[-5pt]
	\text{\tiny co/lax}\\[-5pt]
	\text{\tiny strict}
	\end{rcases*}
	\text{transf.}}\\
	\text{modifications} & \text{modifications}
}
\]
Questa costruzione è una versione lasca e 2-dimensionale della costruzione di Grothendieck (vedi anche \cite[Appendix A]{groth2011monoidal}); difatti il 2-funtore $X$ viene mandato nel 2-funtore $U\mapsto U\eent X$, dove $u\eent X$ è la bicategoria con morfismo tipo $[\xi,\omega,w]\colon (x,S,u)\to (x', S', u')$
\[
\xymatrix{
	\overset{x\in XS}S\ar[rr]^w\ar[dr]_u\ar@<10pt>@{~>}[rr]^\xi && \overset{x'\in XS'}{S'}\ar[dl]^{u'}\\
	& U \ar@{}[u]|{\Swarrow\omega}
}
\]
dove $\xi \colon x \to X(w)x'$, e 2-celle le ovvie $\theta : w \To w'$.
\end{prg}
\begin{prg}
Definiamo un modo di cambiare la varianza di un funtore $(X \colon \C^\opp \to \Cat) \rightsquigarrow (\sharp X \colon \C^\co \to \Cat)$: lo facciamo mediante la composizione
\[
\sharp \colon 
\xymatrix@C=1.5cm{
	[\C^\opp, \Cat] \ar[r]^\eent & [\C^\co,\cate{Hom}] \ar[r]^{[\C^\co,(\pi_0)_*]} & [\C^\co, \Cat]\\
}
\]
Il \emph{morfismo cooperativo} è definito come questa composizione, se $\pi_0 \colon \cate{Hom}\to \Cat$ manda una bicategoria  $\K$ nella categoria che ha gli stessi oggetti e per hom-sets $\pi_0\K(x,y)$.
\end{prg}
\begin{prg}
Esiste una proprietà universale di $\sharp$ legata alla trasformazioni dinaturali lasche. Siano $X : \C^\opp\to \B$, $Z : \C^\co \to \B$ morfismi di bicategorie, e $d : X \xto{..}Z$ dinaturale, considerando $X,Z$ mutamente dipendenti dalle variabili che non sono esplicitamente menzionate.
\end{prg}
\begin{theorem}
Esiste una trasformazione dinaturale universale $X\xrightarrow[\partial]{..} \sharp X$, la composizione con la quale induce un'equivalenza di categorie
\[
[\C^\co, \Cat](\sharp X, Z)\cong \textsf{DiNat}_\text{c}(X,Z)
\]
valida per ogni $Z\colon \C^\co\to\Cat$ e naturale in essa.
\end{theorem}
\begin{remark}
Il morfismo $\sharp$ è univocamente determinato dalla sua azione sui rappresentabili, ovvero da un funtore $\C \to [\C^\co,\Cat]$, ovvero da un funtore $\C\times \C^\co \to \Cat$.
\fcomment{Mi sembra un buon esercizio quello di determinare questo funtore. Incidentalmente, questo risultato e i seguenti devono potersi riguardare nel giusto setting di calcolo delle cofini lasche come segue:
\begin{itemize}
	\item Esiste un'equivalenza $U\eent X \cong \oint^{S\in\C} XS \times \C(S,U)^\opp$;
	\item Esiste una biestensione di Kan
	\[
		\xymatrix{
		\C \ar[r]\ar[d]& [\C^\co,\Cat]\\
		[\C^\opp,\Cat]\ar@/_1pc/@{.>}[ur]
		}
	\]
	\item Nel caso in cui $X$ sia rappresentabile, $V\eent \yon(U) \cong \oint^{S\in\C}\C(S,U)\times \C(S,V)^\opp$.
\end{itemize}}
\end{remark}
\begin{prg}
L'oggetto $\sharp X$ si identifica al bicolimite di $\sharp\yon$ pesato da $X$ (questa è una sorta di coYoneda lemma, o `Yoneda ninja'): c'è il diagramma
\[
\xymatrix{
[\C^\co,\Cat]&\ar[l]_\sharp [\C^\opp,\Cat]& \ar[l]_-{\yon} \faktor{\C}{\C^\opp}\ar@<5pt>[d]^X\\
&& \Cat
}
\]
E' un fatto interessante che la dimostrazione di questo enunciato è del tutto formale: probabilmente è possibile dedurla dalla commutazione di opportune cofini, dato che dovrebbe essere vero che se $W$ è un peso e $Y$ un diagramma,
\[
\sharp(W \star Y) \cong W \star \sharp Y.
\]
(è essenzialmente la cocontinuità di $\sharp$)
\end{prg}
\section{\cite{street1980fibrations}, §3}
Una bicategoria è \emph{finitamente completa} (in quanto bicategoria) quando ha cotensori con $\due$, biequalizzatori, biprodotti finiti e un oggetto biterminale; per la teoria dei bilimiti vedere \cite{makkai1989accessible}. Una monade di Kock-Z\"oberlein è una 2-monade per la quale tutte componenti dell'unità sono equivalenze; si tratta della versione lasca di una monade idempotente perché ne condivide molte proprietà formali (per esempio, la struttura di algebra per una monade idempotente è unica a meno di equivalenza).

Nel seguito $\K$ è una bicategoria finitamente completa.
\begin{prg}
Uno span in $\K$ è un diagramma di 1-celle
\[
A \xot{u}S \xto{v} B
\]
ed esiste una \emph{bicategoria degli span da $B$ ad $A$ in $\K$} con morfismi e 2-celle tipiche
\[
\vcenter{
	\xymatrix@C=1.5cm{
	&S\ar[dr]^v\ar[dl]_u\ddtwocell_{f'}^f{\omega}&\\
	A \ar@{}[r]|{\Searrow}&& B\ar@{}[l]|\Swarrow\\
	& S'\ar[ur]_{v'}\ar[ul]^{u'} & 
	}
}
\quad
\text{\huge =}
\quad
\vcenter{
	\xymatrix@C=1.5cm{
	&S\ar[dr]^v\ar[dl]_u\ar [dd]^{f'}&\\
	A \ar@{}[r]|{\Searrow}&& B\ar@{}[l]|\Swarrow\\
	& S'\ar[ur]_{v'}\ar[ul]^{u'} & 
	}
}
\]
Con ciò, $\textsf{Span}(B,A)$ è una categoria \emph{arricchita in bicategorie}: questo induce delle coerenze sottintese (perché non rilevanti alla nostra discussione): le composizioni sono infatti definite per $n$-uple di oggetti di $\textsf{Span}$ da 1-celle
\[
\circ_n \colon
\xymatrix{
	**[l] \prod_{i=1}^n \textsf{Span}(A_{i+1},A_i) \ar[r] & **[r] \textsf{Span}(A_n, A_0)
}
\]
mediante opportuni bipullback iterati, e 2-celle
\[
\xymatrix{
\prod_{i=1}^{m_k} \textsf{Span}(A_{i+1},A_i) \ar@/_1pc/[dr]\ar[rr]^{\circ_{m_k}}&& \textsf{Span}(A_{m_k}, A_0)\\ 
& \prod_{j=1}^k \textsf{Span}(A_{m_{j+1}}, A_{m_j}) \ar@/_1pc/[ur]\ar@{}[u]|\Sarrow
}
\]
definite per ogni tupla $n_1,\dots, n_k$, ed $m_k = \sum n_i$. (c'è n teorema di coerenza in merito ma per il momento è irrilevante).
\end{prg}
\begin{prg}
Esistono due embedding $(\firstblank)_*\colon \K^\co \to \textsf{Span}(\K)$ e $(\firstblank)^*\colon \K^\opp \to \textsf{Span}(\K)$, definiti rispettivamente dalle regole
\[
\xymatrix@R=0cm{
{\smat{A & \\ \downarrow & \varphi\\ X & }} \ar@{|->}[r] & {\smat{A &=& A \\ \downarrow & & \\ X &&\varphi^*}}\\
{\smat{A & \\ \downarrow & \varphi\\ X & }} \ar@{|->}[r] & {\smat{A &\to& X \\ || & & \\ A &&\varphi_*}}
}
\]
(In fin dei conti, la categoria degli span e quella dei profuntori non si comportano in maniera molto diversa: nel linguaggio sviluppato sotto, \cite[2.8]{street1981conspectus} dimostra che esiste una biequivalenza $\text{Fib}(B,A)\simeq \hom(A^\opp\times B,\Cat)$, e mediante questa biequivalenza il bipullback $E\circ_1 E'$ corrisponde ad una bicofine di funtori $T_E, T_{E'}$, $T_{E\circ E'} = \int^b T(\firstblank,b)\times T(b,\secondblank)$.)
\end{prg}
\begin{prg}
Consideriamo uno span $A \xot{u} S \xto{v} B$ e la composizione
\[
\xymatrix{
S(a,b)\ar[d]\ar[r] & P\ar[d]\ar[r] & X \ar[d]^a\\ 
Q \ar[d]\ar[r]& S\ar[d]_v\ar[r]^u & A \\ 
Y \ar[r]_b & B
}
\]
La definizione di fibrazione cattura condizioni sufficienti affinché l'oggetto denotato come $S(a,b)$ sia funtoriale in $a,b$.

Tutto ruota attorno al diagramma
\[
\xymatrix{
\tre & \due \ar[r]|t \ar[l]_m & \uno \ar@<-6pt>[l]_{d_0}\ar@<6pt>[l]^{d_1}
}
\]
ancora una volta, e siccome cotensorizzare con $A$ è un funtore controvariante $\firstblank \pitchfork A \colon \Cat^\opp\to \K$, otteniamo uno span
\[
\xymatrix{
	&**[l] \due\pitchfork A = \H A \ar[dr]^{\hat d_1}\ar[dl]_{\hat d_0}\\
	\uno \pitchfork A && \uno \pitchfork A
}
\]
e per generiche 1-celle $a : X \to A, b  :Y \to A$ definiamo lo span $\H A(a,b)$ dal diagramma 
\[
\xymatrix{
\H A(a,b) \ar@/^2pc/[rr]^{\tilde d_0} \ar@/_2pc/[dd]_{\tilde d_1} \ar[r]\ar[d] & Q \ar[r]\ar[d]& Y\ar[d]^b \\ 
P \ar[r]\ar[d]& \due\pitchfork A \ar[r]^{\hat d_1}\ar[d]_{\hat d_0}& A \\
X \ar[r]_a & A
}
\]
Questo è chiamato \emph{bicomma} della coppia $(a,b)$. (La ragione è che istanziato in $\Cat$ questo è esattamente l'oggetto comma $(a/b)$).

Il morfismo canonico $\theta \colon \H A(a,b) \to \due\pitchfork A$ corrisponde, per aggiunzione, a un morfismo $\due \to [\H A(a,b),A]$ con source $a\circ \tilde{d}_0$ e target $b\circ \tilde{d}_1$:
\[
\xymatrix{
\H A(a,b) \ar[r]\ar[d]& Y\ar[d]^b \\ 
X \ar[r]_a \ar@{}[ur]|{\Nearrow\lambda} & A
}
\]
In tal senso, $(\H A(a,b)\lambda)$ ha una proprietà universale; è esattamente l'oggetto universale che appare in tale quadrato (ed è definito a meno di equivalenza in $\K$).
\end{prg}
\begin{prg}
Usando tutta la struttura del diagramma
\[
\xymatrix{
\tre & \due \ar[r]|t \ar[l]_m & \uno \ar@<-6pt>[l]_{d_0}\ar@<6pt>[l]^{d_1}
}
\]
otteniamo una KZ-monade in $\textsf{Span}(\K)(A,A)$:
\begin{itemize}
	\item la moltiplicazione risulta dal diagramma
	\[
	\xymatrix{
& \tre \pitchfork A\ar[dr]^{\hat d_2}\ar[dl]_{\hat d_0}\ar[dd]^{\hat m} & \\ 
A && A \\ 
& \ar[ur]_{\hat d_1}\ar[ul]^{\hat d_0}\due \pitchfork A
}
	\]
	prendendo per buono il fatto che $\H A \mathbin{\circ_2} \H A \cong \tre\pitchfork A$; in $\Cat$ questo ammonta esattamente alla composizione $\C_1\times_{\C_0}\C_1 \to \C_1$.
	\item L'unità risulta dal diagramma
	\[	\xymatrix{
& \due \pitchfork A\ar[dr]^{\hat d_1}\ar[dl]_{\hat d_0} & \\ 
A && A \\ 
& \ar@{=}[ur] \ar@{=}[ul]  A\ar[uu]_{\hat t}
}\]
	ottenuto dall'immagine della forcella riflessiva $\uno \rightrightarrows \due \to \uno$.
\end{itemize}
\end{prg}
\begin{prg}
Esistono ora due morfismi di bicategorie
\begin{gather*}
\underline{\bDelta} \xto{\qquad \H \text{A} \qquad} \textsf{Span}(\K)(A,A)\\
\underline{\bDelta} \xto{\qquad \H \text{B} \qquad} \textsf{Span}(\K)(B,B)
\end{gather*}
che inducono dottrine
\begin{gather*}
\underline{\bDelta} \xto{\qquad \H \text{A} \qquad} \textsf{Span}(\K)(A,A) \xto{\quad \textsf{Span}(B,\firstblank)\quad} [\textsf{Span}(B,A),\textsf{Span}(B,A)] \\
\underline{\bDelta} \xto{\qquad \H \text{B} \qquad} \textsf{Span}(\K)(B,B) \xto{\quad \textsf{Span}(\firstblank,A)\quad} [\textsf{Span}(B,A),\textsf{Span}(B,A)]
\end{gather*}
dove $\textsf{Span}(B,\firstblank), \textsf{Span}(\firstblank,A)$ sono opportuni mate della composizione $\circ_2$.
\end{prg}
\begin{prg}
Usando la stessa tecnica con $\circ_3 \colon \textsf{Span}(B,B)\times \textsf{Span}(A,A) \to [\textsf{Span}(B,A), \textsf{Span}(B,A)]$ si ottiene una dottrina (che risulta dalla composizione di $L,R$ mediante una opportuna regola distributiva)
\[
M : \underline{\bDelta} \xto{\left[\begin{smallmatrix} \H A \\ \H B\end{smallmatrix}\right]} \textsf{Span}(A,A)\times \textsf{Span}(B,B) \to [\textsf{Span}(B,A), \textsf{Span}(B,A)]
\]
In queste notazioni, definiamo le
\[
\begin{rcases*}
\text{\tiny fibrazioni sinistre}\\[-6pt]
\text{\tiny fibrazioni destre}\\[-6pt]
\text{\tiny fibrazioni}
\end{rcases*}
\text{da $B$ ad $A$ come le}
\begin{cases}
\text{\tiny $L$-algebre}\\[-6pt]
\text{\tiny $R$-algebre}\\[-6pt]
\text{\tiny $M$-algebre}
\end{cases}
\]
\end{prg}
\def\bipi{\mathbin{\Pi\!\!\Pi}}
Ora, una caratterizzazione meno astratta di questi aggeggi.
\begin{prg}
Ricorda la definizione di oggetto comma, e di oggetto iso-comma in $\Cat$. Denotiamo
\[
\xymatrix{
	A \bipi_C B\ar[r]\ar[d] & A \ar[d]\\
	B \ar[r] & C
}
\]
lo pseudo-pullback di un cospan $A \xto{f} C \xot{g}B$. Denotiamo poi $aEb := aE \bipi_E Eb$
\[
\xymatrix{
	aEb \ar[r]\ar[d]& Eb \ar[r]\ar[d]& 1 \ar[d]^b \\
	aE\ar[r]\ar[d] & E \ar[r]\ar[d]& B\\
	1 \ar[r]_a & A
}
\]
\end{prg}
\begin{prg}
La definizione di fibrazione nasce dalla volontà di rendere pseudofuntoriale l'assegnazione $(a,b)\mapsto aEb$: fissiamo una volta per tutte uno span $\mathfrak S$, contenente $A \xot{p} E \xto{q}B$, e definiamo
\begin{itemize}
	\item un morfismo $\chi : e' \to e$ \emph{cartesiano} se il quadrato
	\[
		\xymatrix{
		E(e'',e')\ar[r]\ar[d] & E(e'',e)\ar[d]\\
		A(pe'',pe') \ar[r] & A(pe'',pe)
		}
	\]
	è un pullback in $\Set$;
	\item lo span $\mathfrak S$ una \emph{fibrazione sinistra} se per ogni oggetto $e\in E$ e morfismo $\alpha : a' \to pe$ in $A$ esiste un morfismo cartesiano $\chi^\alpha : e^\alpha \to e$ in $E$ e un isomorfismo $\gamma' : a' \cong pe'$ tale che $\alpha = p(\chi)\circ \gamma'$. 
\end{itemize}
Una scelta di $\chi^\alpha$ (e implicitamente dell'oggetto $e^\alpha$) si dice \emph{clivaggio}\footnote{Il clivaggio (dal francese \emph{clivage}, derivato dall'olandese \emph{klieven} ``fendere'') è la naturale tendenza di determinate strutture a separarsi, per la presenza di un'interfaccia tra due materiali diversi.} per $\alpha$. Si osservi che dati due clivaggi $\chi^\alpha_1,\chi^\alpha_2$ c'è un unico isomorfismo $\chi^\alpha_1 \to \chi^\alpha_2$.
\end{prg}
\begin{prg}
Ora, l'idea per indurre un funtore
\[
\firstblank E  : A^\opp \to \Cat_{/\!\!/B}
\]
che manda $a\mapsto aE$ è quella di precomporre un oggetto $a\xto{\sim} pe$ con un morfismo $a'\to a$ per ottenere $a'\to a\to pe$; del resto, dato che $a'E$ è un iso-comma oggetto, questa freccia dovrebbe essere invertibile. La condizione di fibrazione serve a poter sostituire, in maniera sufficientemente canonica, $a'\to a\to pe$ ad un oggetto in $a'E$: si ha infatti che esiste
\[
\xymatrix{
	p(e^{\gamma\alpha}) \ar[d]& \ar[l]^-{\gamma'}_-\sim a' \ar[d]^\alpha \\
	pe & a\ar[l]^\sim
}
\]
La funtorialità di questa corrispondenza viene ancora dalla condizione di fibrazione:
\begin{itemize}
	\item Per la composizione $a'' \xto{\alpha'} a' \xto{\alpha} a$ si ha che \dots
	\item Per l'identità $1_a : a \to a$ si ha che \dots
\end{itemize}
\end{prg}
\begin{prg}
In effetti si è descritta una biequivalenza di bicategorie $\text{LFib}(B,A) \cong \hom(A^\opp, \Cat_{/\!\!/B})$ definita mandando $\mathfrak S$ in $\firstblank E$.
\end{prg}
\bibliography{allofthem}{}
\bibliographystyle{amsalpha}
\hrulefill 
\end{document}