\setlist[1]{itemsep=0pt}

\renewcommand{\textbf}[1]{\text{\fontseries{b}\selectfont{\upshape #1}}}

\newcommand{\arXivPreprint}[1]{arXiv preprint \href{http://arxiv.org/abs/#1}{arXiv:#1}}

\newtheoremstyle{reference}%
   {}
   {}
   {}                      % Font del testo
   {}                      % Rientro margini
   {\bfseries}             % Font del titolo dell'ambiente
   {:}                     % Punteggiatura dopo "Teorema"\"Definizione"
   {.2em}                  % Spazio tra titolo e testo.
   {\thmname{#1}           % #1 : Definizione\Teorema\ecc
    \thmnumber{#2}         % #2 : Contatore
    \thmnote{{\sc [#3]}}}  % #3 : Testo tra "[" e "]"

\theoremstyle{reference}
  \newtheorem{theorem}{Theorem}[section]
  \newtheorem{claim}[theorem]{Claim}
  \newtheorem{conjecture}[theorem]{Conjecture}
  \newtheorem{conjec}[theorem]{Conjecture}
  \newtheorem{corollary}[theorem]{Corollary}
  \newtheorem{counterex}[theorem]{Counterexample}
  \newtheorem{definition}[theorem]{Definition}
  \newtheorem{example}[theorem]{Example}
  \newtheorem{exercise}[theorem]{Exercise}
  \newtheorem{lemma}[theorem]{Lemma}
  \newtheorem{notat}[theorem]{Notation}
  \newtheorem{proposition}[theorem]{Proposition}
  \newtheorem{question}[theorem]{Question}
  \newtheorem{remark}[theorem]{Remark}
  \newtheorem{scholium}[theorem]{Scholium}
  \newtheorem{setting}[theorem]{Setting}
  % starred
  \newtheorem*{claim*}{Claim}
  \newtheorem*{conjec*}{Conjecture}
  \newtheorem*{corollary*}{Corollary}
  \newtheorem*{counterex*}{Counterexample}
  \newtheorem*{definition*}{Definition}
  \newtheorem*{example*}{Example}
  \newtheorem*{exercise*}{Exercise}
  \newtheorem*{lemma*}{Lemma}
  \newtheorem*{notat*}{Notation}
  \newtheorem*{proposition*}{Proposition}
  \newtheorem*{question*}{Question}
  \newtheorem*{remark*}{Remark}
  \newtheorem*{scholium*}{Scholium}
  \newtheorem*{setting*}{Setting}
  \newtheorem*{theorem*}{Theorem}

% ########################################
% ALCUNE COSE VANNO SEMPRE IN FONDO

\hypersetup{%
  pdftoolbar=   true,
  pdfmenubar=   true,
  pdffitwindow= true,
  % pdftitle=     {},
  % pdfauthor=    {},
  colorlinks=   true,
  linkcolor=    black,
  citecolor=    blue!40!black}


% \cite e \ref in bold
\makeatletter
  \def\@cite#1#2{[\textbf{#1}\if@tempswa , #2\fi]}
  \def\@biblabel#1{[\textsf{#1}]}
\makeatother
\providecommand{\refbf}[1]{\textbf{\ref{#1}}}

% abbreviazioni
\providecommand{\abbrv}[1]{#1.\@\xspace}
  \providecommand{\ie}{\abbrv{i.e}}
  \providecommand{\etc}{\abbrv{etc}}
  \providecommand{\prof}{\abbrv{prof}}
  \providecommand{\viz}{\abbrv{viz}}
  \providecommand{\eg}{\abbrv{e.g}}
  \providecommand{\achap}{\abbrv{Ch}}
  \providecommand{\adef}{\abbrv{Def}}
  \providecommand{\acor}{\abbrv{Cor}}
  \providecommand{\aprop}{\abbrv{Prop}}
  \providecommand{\athm}{\abbrv{Thm}}


\newlength{\seplen}
\setlength{\seplen}{5pt}
%
\newlength{\sepwid}
\setlength{\sepwid}{.4pt}
%
\def\firstblank{\,\rule{\seplen}{\sepwid}\,}
\def\secondblank{\firstblank\llap{\raisebox{2pt}{\firstblank}}}

\setcounter{tocdepth}{1}

\newcommand{\mailto}[1]{\href{mailto:#1}{\sf #1}}
\newcommand{\comment}[1]{\todo[inline,caption={}]{\textbf{Someone says} : #1}}

% REMOVE THIS TO GET BACK AMSART HEADERS ========
\pagestyle{fancy}
\chead{\texttt{\tiny \color{red}TeX 3.14159265 (TeX Live 2016)
 : \today \hfill\currenttime}}
\lhead{}
\rhead{}
\cfoot{}%{\texttt{\tiny \color{red}TeX 3.14159265 (TeX Live 2016)
 : \today \hfill\currenttime}}
\lfoot[\thepage]{}
\rfoot[]{\thepage}